\section{Введение}
Развитие арабской математики началось в 7 в. нашей эры, как раз  в эпоху возникновения религии  ислама. Она выросла из многочисленных  задач, поставленных торговлей, архитектурой, астрономией, географией, оптикой, и глубоко сочетала  в себе стремление решить эти практические задачи и напряженную  теоретическую работу.

Арабские  математики добились решающих  достижений и сделали ряд неоспоримых   открытий в области разработки  алгебраического исчисления, как  абстрактного, так и практического,  становления теории уравнений, алгоритмических методов на стыке алгебры и арифметики.

В развитии арабской математики  можно  различить два периода:  прежде всего  усвоение в  VII и VIII вв. греческого и восточного  наследия. Багдад был первым крупным  научным центром в правления Аль-Мансура (754-775) и Гарун ал-Рашида (786-809). Там было большое количество  библиотек, и изготовлялось много  копий научных трудов. Переводились  труды античной Греции (Евклид, Архимед,  Аполлоний, Герон, Птолемей, Диофант), изучались также труды из Индии, Персии и Месопотамии. 

Но  к IX в. сформировалась настоящая  собственная математическая культура, и новые работы вышли за  рамки, определенные эллинским  математическим наследием.

Первым  знаменитым ученым багдадской  школы был Мухаммед Аль-Хорезми, деятельность которого протекала в первой половине IX в. Он входил в группу математиков и астрономов, которые работали в Доме мудрости, своего рода академии, основанной в Багдаде в правление ал-Маммуна (813-833). Сохранились пять работ ал-Хорезми, частично переработанные, из которых два трактата об арифметике и алгебре оказали решающее воздействие на дальнейшее развитие математики.

Его  трактат об арифметике известен  только в латинском варианте XIII в., который, без сомнения, не является  точным переводом. Его можно  было бы озаглавить «Книга  о сложении и вычитании на  основе индийского исчисления».  Это, во всяком случае, первая  книга, в которой изложены десятичная  система счисления и операции, выполняемые в этой системе,  включая умножение и деление.  В частности, там использовался  маленький кружочек, выполнявший  функции нуля. Ал-Хорезми объяснял, как произносить числа, используя понятия единицы, десятка, сотни, тысячи, тысячи тысяч…, которые он определил. Но форма использованных ал-Хорезми цифр неизвестна, возможно, это были арабские буквы или арабские цифры Востока.
 
О  происхождении  арабских цифр  стоит сказать отдельно.   Арабские  цифры — традиционное название  десяти математических знаков: 0, 1, 2, 3, 4, 5, 6, 7, 8, 9, с помощью которых  по десятичной системе счисления  записываются любые числа. Эти  цифры возникли в Индии (не  позднее V в.), в Европе стали  известны в Х-ХIII вв. по арабским сочинениям (отсюда название).  

Интересные  факты: Ряд интересных математических задач, стимулировавших развитие сферической геометрии и астрономии, поставила перед математикой и сама религия ислама. Это задача о расчёте лунного календаря, об определении точного времени для совершения намаза, а также об определении киблы — точного направления на Мекку.  
