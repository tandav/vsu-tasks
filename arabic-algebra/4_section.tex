\section{Источники арабской математики}
Важно взглянуть на используемые источники Аль-Хорезми, чтобы определить влияние греческой математики на его алгебру. Также важно обсудить современные источники Аль-Хорезми, или как современные ученые знают о работах Аль-Хорезми.

\subsection{Источники Аль-Хорезми}
Естественно начать с изучения источников, используемых Аль-Хорезми. Существует три теории, посвященные источникам, используемым Аль-Хорезми в начале алгебры; к ним относятся теории, в которых он использовал классические греческие источники, индуистские источники, популярные сирийско-персидско-ивритские математические труды

Согласно Тумеру, как описано в История альгебры Ван дер Вардена: от Fль-Хорезми до Эмми Нётер\cite{vanderwaer}, как индуистская, так и греческая алгебра вышли далеко за рамки элементарной стадии работы Аль-Хорезми. Доказательства, включенные в его работу, не имеют существенных различий в известных произведениях любой культуры. Например, его доказательства методов решения квадратичных уравнений существенно отличаются от доказательств, найденных в Элементах Евклида. Кроме того, сохранившийся алгебраический трактат греческой культуры, написанный Диофантом, развился в направлении символьного представления, в то время как  трактат Аль-Хорезми - риторический. В этом отношении работа ль-Хорезми аналогична работе санскритских алгебраических произведений. По этим причинам маловероятно, что на Аль-Хорезми значительно  повлияла классическая греческая математика.

Аль-Хорезми написал трактат об индусских цифрах, и двое из его приближений числа $\pi$ были найдены в индуистских источниках, поддерживая теорию о том, что на работу Аль-Хоризмами влияли индуистские источники. Аль-Хорезми далее ссылался на свои источники в своем разделе «Мероопределение / Мензурация» в своей Алгебре:

\begin{displayquote}
Однако у математиков есть два других правила. Одно из них: умножить диаметр на себя, затем на десять, а затем взять квадратный корень произведения. Корень дает длину окружности.

Другое правило используется среди астрономов и гласит: умножьте диаметр на шестьдесят две тысячи восемьсот тридцать два, а затем разделите его на двадцать тысяч. Частное дает длину окружности.
\end{displayquote}

Первое правило ($p = \sqrt{10d^2}d$) найдено в главе 12 \textit{Брахмашфутасиддханта} Брахмагупты, поддерживая теорию о том, что Аль-Хорезми был знаком с индуистскими алгебраическими трактатами. Аль-Хорезми
приписал второе правило ($p = \frac{62832}{20000}d$) к астрономам, и уравнение найдено в \textit{Aryabhatiya} индуистского астронома \textit{Aryabhata} с начала шестого века нашей эры. Поскольку Аль-Хорезми использовал как персидский, так и индуистский источники для составления своих астрономических таблиц, вполне вероятно, что он также получил свои оценки $\pi$ из этих источников.

Третья теория утверждает, что на работы Аль-Хорезми повлияла местная сирийско-персидско-ивритская традиция. Это подтверждается тесной связью между геометрией Аль-Хорезми и ивритским трактатом Мишнат ха-Миддо. Эту теорию поддержал также Соломон Гандз, редактор Мишнат ха-Миддо. Он обсуждает свой взгляд на Аль-Хорезми как «антагонист греческого влияния», заявляя, что Аль-Хорезми никогда не упоминает своего коллегу Аль-Хаджадж ибн Юсуф ибн Матар. Аль-Хаджадж посвятил свою жизнь переводу греческой математической, философской и научной работы на арабский язык. Однако Аль-Хорезми не относится к Евклиду и его геометрии при написании своего собственного геометрического трактата; Кроме того, Аль-Хорезми подчеркивает свою цель написать практический алгебраический трактат в противоречии с греческой теоретической математикой в предисловии к его алгебраическому трактату. Из-за этого Соломан Ганд утверждает:

\begin{displayquote}
Аль-Хорезми представляется нам не как ученик греков, а, наоборот, как антагонист аль-Хаджаджа и греческой школы, как представитель местных народных наук. В Академии Багдада [Дом Мудрости] Аль-Хорезми представлял скорее реакцию против введения греческой математики. Его Алгебра впечатляет нас как протест против перевода Евклида и против всей тенденции принятия греческих наук.
\end{displayquote}

Кажется вероятным, что хотя Аль-Хорезми не был подверженвлиянию греческой математики, комбинация второй и третьей теорий может наилучшим образом описать влияние на алгебру Аль-Хорезми и геометрию. Как индуистские источники, так и популярная математика сирийско-персидско-еврейских источников, похоже, присутствуют в работе Аль-Хорезми, как видно из его использования Брахмашфутасиддханты, Арябатии и Мишнат ха-Миддота, а также из-за отсутствия сходства алгебры и геометрии Аль-Хорезми и греческой алгебры и геометрии.

\subsection{Современные источники Аль-Хорезми}
Не менее важным в нашем обсуждении работы Аль-Хорезми и его источников являются источники, которые современные историки и математики используют, чтобы знать его работу. Мы знаем о средневековой исламской математике в основном через арабские документы; математические трактаты средневековых арабских математиков можно найти в библиотеках и частных коллекциях по всему миру. Эти коллекции в основном встречаются в странах, которые когда-то были частью исламского средневекового мира, но значительные коллекции существуют также в Англии, Франции, Германии и России: все страны, которые были колониальными державами в исламском мире.

Большинство из этих трактатов являются прозаическими композициями, но могут включать в себя таблицы чисел, некоторые из которых содержат сотни тысяч записей. Эти таблицы были рассчитаны главным образом для астрономических целей и почти никогда не включают в себя объяснения того, как были вычислены числа или записи в ячейках таблицы. Физические артефакты также предоставляют важные источники исламской математики, такие как математические и астрономические инструменты. Примерами этих артефактов являются три карты мира в виде круговых дисков. Они позволили пользователям найти направление Мекки, вращая линейку вокруг центра диска. Прозаические трактаты, таблицы и инструменты позволяют современным ученым изучать средневековую исламскую математику.

Хороший отрывок (переведенный на английский язык) Аль-Джабра Аль-Хорезми и его трактат о индуистских цифрах можно найти в главе Дж. Леннарта Берггрена «Математика в средневековом исламе» в \textit{«Математике Египта», Месопотамии, Китая, Индии , и Ислама: «Справочник»}, Издательство Принстонского университета, 2007.
