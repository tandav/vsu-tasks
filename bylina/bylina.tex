\documentclass[a4paper, 12pt]{article}
\usepackage[utf8]{inputenc}
\usepackage[russian]{babel}
\usepackage{literat}
\usepackage[T2A]{fontenc}
\usepackage{verse}
\usepackage{geometry}

\setlength\parindent{0pt}
\settowidth{\versewidth}{Во славном во городе во Киеве}
\geometry{left=1cm, right=1cm, top=1cm, bottom=1.5cm}

\title{Анализ былины\\\textbf{\scshape\Huge<<Наезд на богатырскую заставу и бой сына Ильи Муромца с отцом>>}}
\author{\Large{\scshapeАлександр Родионов}\\ФКН, МКН, 1к (магистратура)}
\date{25 мая 2018}

\begin{document}

\maketitle

\section{Жанр}
Данная былина относится к героическому, богатырскому типу былин. Основная тема данной былины - поединки нахвальщика с Добрынюшкой Микитичем и с Ильей Муромцем (впоследсвии оказавшимся отцом нахвальщика).


\section{Тип}
Данная былина относится к Киевскому циклу. На это указывают первые строчки былины так как 
\begin{verse}[\versewidth]
\itshape
Во славном во городе во Киеве\\
У славного князя Владимёра
\end{verse}


Действие былины происходит в Киеве во времена правления князя Владимира.

\section{Анализ персонажей}
\subsection{Нахвальщик}
Молодой богатырь. 12 лет. Сильный и смелый персонаж. Вот как он вызывает людей на поединок:
\begin{verse}[\versewidth]
\itshape
Уж ты давай-ко мне поединьщика\\
Во чистом поле да поборотися,\\
Могучим плечам да росходитися,\\
Богатырскому серцу розгоретися,\\
Молодецкой крови розкипетися!
\end{verse}


\subsection{Федосья Тимофеёвна}
Матушка нахвальщика. Любит своего ребенка:
\begin{verse}[\versewidth]
\itshape
<<Уж ты ой еси, мое чадо милое!>>
\end{verse}

\subsection{Илья Муромец}
Отец нахвальщика, богатырь, казак. Атаман (военачальник) богатырской заставы.
\subsection{Добрыня Микитич}
Богатырь, казак. Ясаул (помощник военачальника) богатырской заставы. В Данной былине Добрыня Микитич обладает трусливым характером, так как испугался боя с нахвальщиком:
\settowidth{\versewidth}{А-й к семи борцам, к семи удальцам, к се}
\begin{verse}[\versewidth]
\itshape
Он приехал нонь да во чисто поле,\\
Посмотрял он вить в полотенце долговидное:\\
Как там ездит нахвальщик великие,\\
Он кидает палицу вверх да под облако, —\\
Не спущает палицу да на сыру землю,\\!

Принимает палицу да в едину руку.\\
Тут у Добрыни конь на карачу пал,\\
Тут Добрынюшка нонь назад подрал\\
Он к той к заставушке нонь ко киевской,\\
А-й к семи борцам, к семи удальцам, к семи богатырям,\\!

Он к семи поленицам розудалыех.\\
«Сколько я в чистое полё нонь не ежживал,\\
Такой беды я больше да не видывал\ldots\\
\end{verse}


\section{Главная мысль и краткий сюжет}
Основной сюжет былины - казус. Нахвальщик устраивает поединок со своим отцом Ильей Муромцем и чуть не убивает его. Но в конце они узнают друг друга.

\begin{itemize}
    \item Мать провожает нахвальщика в чисто поле, она вероятно знает о том, что нахвальщик может встретить в чистом поле отца и просит не сражаться с ним:
    \begin{verse}[\versewidth]
        \itshape
            Ты пойедёшь нонь да во чисто полё;\\
            Уж ты стретишься во чистом поли ты со старыём,\\
            Ты со старыем да со седатыем;\\
            Ты не бейся с им да не ранийся,\\
            Соходи с коня да ниско кланейсе».
        \end{verse}
    \item Нахвальщик заезжает на богатырскую заставу и говорит что хочет поединка с кем-нибудь. 
    \item В чистом поле состоится поединок с добрыней микитичем. Добрыня испугался и сбежал с поля боя
    \item Нахвальщик возвращается на богатырскую заставу и продолжает требовать поединка
    \item Состоится поединок с Ильей Муромцем
    \item Нахвальщик почти побеждает Илью Муромца и хочет зарезать его:
        \begin{verse}[\versewidth]
        \itshape
            Выходили оне на рукопашный бой.\\
            Уж он брал его да за седы власа,\\
            Он кидал его да во сыру землю;\\
            Он скочил к ему да на белы груди;\\
            Доставал веть он свой булатный нож,\\
            Хотел пороть грудь да белые,\\
            Вынимал хочёт серце да с печенью.
        \end{verse}

    \item Илья Муромец молится и у него получается вырваться. 
    \item Илья Муромец спрашивает о родственниках нахвальщика 
    \item Нахвальщик отвечает
        \begin{verse}[\versewidth]
        \itshape
            «Отца-то я нонь не имаю,\\
            Еще мать у меня да Тимофеёвна».
        \end{verse}


    \item Илья Муромец понимает, что нахвальщик - это его сын, он обнимает его, целует и говорит
        \begin{verse}[\versewidth]
        \itshape
            «Уж ты ой еси, да чадо милоё,\\
            Чадо милоё, дитя любимоё!\\
            Еще ты нонь в чистом поли нонь засеяно;\\
            Ты свези да поклон да от Ильи Муромца!»
        \end{verse}
    \item Нахвальщик едет к матери, передает поклон от Ильи Муромца 
    \item Нахвальщик возвращается на богатырскую заставу где ему поют славу
\end{itemize}

\section{Художественные приемы}

\subsection{Эпитеты}
\begin{itemize}
    \item бурушка трехлеточка
    \item полотенцо долговидноё
    \item синё море
    \item сыру землю
    \item Уж он бил коня да по крутым бедрам
    \item во чистом поли
    \item белы груди
    \item сахарны уста
    \item Отрубил матери по плеч да голову (низкий поклон)
\end{itemize}

\subsection{Слова с уменьшительно-ласкательными суффиксами}
\begin{itemize}
    \item семь \textbf{полениц}
    \item \textbf{Добрынюшка, Алёшенька, Никитушка}
    \item Что ёт \textbf{камешка} было от Латыря,
    \item \textbf{нахвальщичок}
    \item Ище брал он себе бурушка \textbf{трехлеточка}
    \item Уж он клал ему потнички на \textbf{потнички};
    \item А-й на потнички накладывает \textbf{войлочки};
    \item А-й на войлочки накладывал \textbf{седелышко} церкасскоё
    \item Уж он брал себе \textbf{сабельку} да вострую.
    \item Выходила к нему \textbf{матушка}
    \item Он приехал к той \textbf{заставушке} нонь ко киевской,
    \item \ldots
\end{itemize}
и другие
\subsection{Повторы}
\subsubsection{Перечисление родственников}
\begin{verse}[\versewidth]
\itshape
А-й во-третьих был Алёшенька Попович-то,\\
А в-четвёртых был Никитушка Романович,\\
А во-пьятых был гость торговые,\\
А-й шестой и седьмой Кострюк, Демрюк, дети боярские.
\end{verse}

\subsubsection{Выбор оружия}
\begin{verse}[\versewidth]
\itshape
Уж он брал себе палицу боёвую,\\
Уж он брал себе копьё да бургоминскоё,\\
Уж он брал себе сабельку да вострую,\\
\end{verse}

\subsubsection{3 съезда на лошадях в сцене поединка с нахвальщика с Ильей Муромцем}
\begin{verse}[\versewidth]
\itshape
Они съехались да поздоровались.\\
Они начали съежжатьсе на двенадцать верст.\\
Первой раз съежжалисе на палицы да на боёвые:\\
Он-не друг другу богатыря всё ударили —\\
У них палицы нонь да по яблучкам да прочь отпадали,\\
Они друг друга богатыри всё не ранили.\\!

Они съехались второй раз на копья на бургоминские:\\
Оне друг друг богатыри всё да ударили,\\
У их копья да по яблучкам прочь отпадали —\\
Оне друг друг богатыри всё не ранили.\\!

Оне выежжали нонь во третей раз,\\
Оне съежжалисе на сабельки да вострые:\\
Оне друг друга богатыри всё да ударили —\\
У их сабельки по ручкам да прочь отпадали.
\end{verse}


\subsubsection{Илья Муромец 2 раза спрашивал нахвальщика про его родственников}
\begin{verse}[\versewidth]
\itshape
Он кидал его да о сыру землю;\\
Он спросил его да роду-племени,\\
Роду-племени да отца с матерью.\\
«Что ты скаживашь роду-племени,\\
Роду-племени да отца с матерью?\\
Отца с матерью я да не имаю».\\!

Уж он брал ещё его да за желты кудри;\\
Он кидал его да о сыру землю;\\
Он спросил его да роду-племени,\\
Роду-племени да отца с матерью.\\
«Отца-то я нонь не имаю,\\
Еще мать у меня да Тимофеёвна».
\end{verse}

\subsubsection{Описание пира на богатырской заставе}
\begin{verse}[\versewidth]
\itshape
Как он за тема за столами да дубовыма,\\
Он за тема скатертями частобраными,\\
Он за тема за яствами за сахарними,\\
Он за тема питиями виноградными.
\end{verse}


\subsection{Общие места}
В былине присутствуют несколько повторяющихся блоков

\subsubsection{Нахвальщик требует поединок (2 раза)}
\settowidth{\versewidth}{Богатырскому серцу розгоретися,}
\begin{verse}[\versewidth]
Уж ты давай-ко мне поединьщика\\
Во чистом поле да поборотися,\\
Могучим плечам да росходитися,\\
Богатырскому серцу розгоретися,\\
Молодецкой крови розкипетися!»
\end{verse}

\subsubsection{Сцена подготовки к поединку (3 раза)}
\settowidth{\versewidth}{А-й на войлочки накладывал седелышко церка}
\begin{verse}[\versewidth]
\itshape
Ище брал он себе бурушка трехлеточка;\\
Уж он клал ему потнички на потнички;\\
А-й на потнички накладывает войлочки;\\
А-й на войлочки накладывал седелышко церкасскоё;\\
Подтягал седло да во двенадцать пряж,\\
Чтобы конь ис-под богатыря вон не выскочил,\\
Чтоб богатыря в чистом поле не оставил бы,\\
Он во тех во лесах да во черниговских,\\
Он на той на площади да богатырские.\\
Он брал себе палицу во сорок пуд,\\
Уж он брал себе копиё да бургоминскоё,\\
Уж он брал себе сабельку да вострую.\\
Видели Добрынюшку сажаючись,\\
А-й не видели да поезжаючись:\\
Еще не пыль в поле, да курева стоит,\\
Еще матушка сыра земля в поле стонучись стонёт.\\
\end{verse}

\subsubsection{На поле перед поединком (2 раза)}
\settowidth{\versewidth}{Посмотрял он вить в полотенце долговидноё:}
\begin{verse}[\versewidth]
Он приехал нонь да во чисто полё,\\
Посмотрял он вить в полотенце долговидноё:\\
Как там ездит нахвальщик великие,\\
Он кидает палицу вверх да под облако, —\\
Не спущает палицу да на сыру землю,\\
Принимает палицу да в едину руку.\\
Тут у Добрыни конь на карачу пал,\\
\end{verse}

\end{document}
