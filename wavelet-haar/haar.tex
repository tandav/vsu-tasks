\pdfmapfile{+literat.map}
\documentclass[a4paper, 12pt]{article}
\usepackage[utf8]{inputenc}
\usepackage[russian]{babel}
\usepackage{literat}
\usepackage[T2A]{fontenc}
% \usepackage{geometry}
\usepackage[left=1cm, right=2.5cm, top=1cm, bottom=1.5cm]{geometry}
\usepackage{graphicx}
% \usepackage{setspace}
\usepackage{blindtext}
\usepackage{multicol}
\usepackage{amsmath}
\graphicspath{ {./images/} }

\DeclareMathOperator{\Haar}{H}
\DeclareMathOperator{\iHaar}{iH}
% \setlength{\columnsep}{1cm}
% \usepackage[export]{adjustbox}
% \setlength\parindent{0pt}

\title{\large{Родионов Александр МКН 1к маг.}}
\date{}
% \author{\Large{\scshapeАлександр Родионов}\\ФКН, МКН, 1к (магистратура)\\}


\begin{document}
\maketitle
\vspace{-2cm}

% \twocolumn

\begin{multicols}{2}

\noindent\textbf{Задача.} Исследовать насколько сильно можно сжать сигнал с помощью вейвлет преобразования Хаара без существенных потерь качества. Последовательность действий: 

\begin{enumerate}
  \item исходный сигнал $f = \sin(x) + 0.15\sin(20x)$
  \item преобразавание Хаара: $h = \Haar(f)$
  \item Обнуление коеффициентов, абсолютное значение которых меньше определенного порога $T$: $$h_{i} = 0, \text{если }\mathopen|h_{i}|< T$$ Подсчет количества выброшенных коеффициентов $K$. Вычесление процента сжатия:  $compressions = \frac{K}{N} \times 100\%$
  \item Обратное преобразавание Хаара: $$g = \iHaar(h)$$
  \item Сравнение $f$ и $g$
\end{enumerate}

Сравнение сигналов будем определять тремя способами: 


\begin{itemize}
  \item Cравнение графиков $f$ и $g$
  \item $RMSE = \sqrt{    \frac{1}{n} \sum_{i=1}^{n}(f_{i} - g_{i})^2}    $
  \item $MAE = \frac{\sum_{i=1}^{n}|f_{i} - g_{i}|}{n}$
\end{itemize}



% \thispagestyle{empty}

\begin{center}
ФЕДЕРАЛЬНОЕ ГОСУДАРСТВЕННОЕ БЮДЖЕТНОЕ ОБРАЗОВАТЕЛЬНОЕ УЧРЕЖДЕНИЕ ВЫСШЕГО ОБРАЗОВАНИЯ\\
<<ВОРОНЕЖСКИЙ ГОСУДАРСТВЕННЫЙ УНИВЕРСИТЕТ>>\\
\vspace{0.25cm}
Факультет компьютерных наук\\
\vspace{0.25cm}
Кафедра цифровых технологий\\

\vspace{1cm}

\textit{Математика и компьютерные науки}\\
\textit{Распределенные системы и искусственный интеллект}\\

\vspace{4cm}

\large{доклад на тему\\}
\Huge{\scshape{Достижения арабских математиков в алгебре}}

\vspace{3cm}


\begin{small}
\begin{tabbing}
ооооооооооооооо \=  ----------------------  \kill
Обучающийся \>  \rule[0mm]{3cm}{0,3mm}  \textit{А.А. Родионов, 1 курс, (магистратура)} \\ 
ооооооооооооооо \=  ----------------------  \kill
Преподаватель\>  \rule[0mm]{3cm}{0,3mm}  \textit{С.А. Скляднев, к. физ-мат. н., доцент }
\end{tabbing}

\vspace{9cm}

\centerline{Воронеж 2018}
\end{small}

\end{center}


% \newpage


% \newpage
% \addcontentsline{toc}{section}{\protect\numberline{}Список использованных источников}%
% \bibliographystyle{plain}
% \bibliography{5_references}
% \printbibliography

\end{multicols}

\vspace{1cm}
% \begin{figure}[h]
\centering
% \includegraphics[scale=1.5]{signals-0_1}
% \includegraphics[scale=1.5]{signals-0_3}
\includegraphics[width=18cm]{signals-0_1}
\includegraphics[width=18cm]{signals-0_3}

Вычислим также значения RMSE и MAE для всех возможных уровней компрессии 0...100\%:
\vspace{2cm}
\includegraphics[width=18cm]{rmse_mae}

% \caption{Example of a parametric plot ($\sin (x), \cos(x), x$)}
% \end{figure}
\end{document}
